\documentclass{article}
\usepackage[cm]{fullpage}
\usepackage{color}
\usepackage{hyperref}
\usepackage{geometry}

\hypersetup{breaklinks=true,%
colorlinks=true,%
linkcolor=cyan,%
urlcolor=MyDarkBlue}

\definecolor{MyDarkBlue}{rgb}{0,0.0,0.45}

%%%%%%%%%%%%%%%%%%%%%%%%%%
% Formatting parameters  %
%%%%%%%%%%%%%%%%%%%%%%%%%%

\newlength{\tabin}
\setlength{\tabin}{1em}
\newlength{\secsep}
\setlength{\secsep}{0.1cm}

\setlength{\parindent}{0in}
\setlength{\parskip}{0in}
\setlength{\itemsep}{0in}
\setlength{\topsep}{0in}
\setlength{\tabcolsep}{0in}

\definecolor{contactgray}{gray}{0.3}
\pagestyle{empty}

%%%%%%%%%%%%%%%%%%%%%%%%%%
%  Template Definitions  %
%%%%%%%%%%%%%%%%%%%%%%%%%%

\newcommand{\lineunder}{\vspace*{-8pt} \\ \hspace*{-6pt} \hrulefill \\ \vspace*{-15pt}}
\newcommand{\name}[1]{\begin{center}\textsc{\Huge#1}\\\end{center}}
\newcommand{\program}[1]{\begin{center}\textsc{#1}\end{center}}
\newcommand{\contact}[1]{\begin{center}\color{contactgray}{\small#1}\end{center}}

\newenvironment{tabbedsection}[1]{
  \begin{list}{}{
      \setlength{\itemsep}{0pt}
      \setlength{\labelsep}{0pt}
      \setlength{\labelwidth}{0pt}
      \setlength{\leftmargin}{\tabin}
      \setlength{\rightmargin}{\tabin}
      \setlength{\listparindent}{0pt}
      \setlength{\parsep}{0pt}
      \setlength{\parskip}{0pt}
      \setlength{\partopsep}{0pt}
      \setlength{\topsep}{#1}
    }
  \item[]
}{\end{list}}

\newenvironment{nospacetabbing}{
    \begin{tabbing}
}{\end{tabbing}\vspace{-1.2em}}

\newenvironment{resume_header}{}{\vspace{0pt}}


\newenvironment{resume_section}[1]{
  \filbreak
  \vspace{2\secsep}
  \textsc{\large#1}
  \lineunder
  \begin{tabbedsection}{\secsep}
}{\end{tabbedsection}}

\newenvironment{resume_subsection}[2][]{
  \textbf{#2} \hfill {\footnotesize #1} \hspace{-4em}
  \begin{tabbedsection}{0.5\secsep}
}{\end{tabbedsection}}

\newenvironment{subitems}{
  \renewcommand{\labelitemi}{-}
  \begin{itemize}
      \setlength{\labelsep}{1em}
}{\end{itemize}}

\newenvironment{resume_employer}[4]{
  \vspace{\secsep}
  \textbf{#1} \\ 
  \indent {\small #2} \hfill {\footnotesize#3 (#4)} \hspace{-4em}
  \begin{tabbedsection}{0pt}
  \begin{subitems}
}{\end{subitems}\end{tabbedsection}}

%%%%%%%%%%%%%%%%%%%%%%%%%%
%     Start Document     %
%%%%%%%%%%%%%%%%%%%%%%%%%%

\begin{document}
\newgeometry{left=0.5in,right=0.5in,top=0.5in,bottom=0.25in}
\begin{resume_header}
\name{William Liao}
\program{Mathematics \& Computer Science @ UMD}
\contact{william.q.liao@gmail.com \hspace{2cm} +1 (610) 999-0818}
%\hspace{2cm}833 Williamsburg Blvd., Downingtown, PA 19335, United States
\end{resume_header}

\begin{resume_section}{Education}
	
	\begin{resume_subsection}[College Park, MD (Aug. 2020 -- )]{University of Maryland, College Park}
    	\begin{subitems}
        \item Bachelor of Science - BS, Mathematics (Expected May 2024) \hspace*{0pt}\hfill GPA: 3.94 / 4.00 
        \begin{subitems}
            \item Minor in Computational Finance
        \end{subitems} 
      	\item Bachelor of Science - BS, Computer Science (Expected May 2024)     
      	\item Banneker/Key (Full Ride) Scholarship Recipient
      	% \item GPA: 3.93
      	\item Relevant Coursework - Probability Theory, Theory \& Methods of Statistics, 
      	Financial Markets and Financial Datasets, Portfolio Management, Financial Econometrics, Partial Differential Equations,
        Machine Learning, Numerical Analysis, Design and Analysis of Algorithms, Linear Algebra, Calculus.   
        
     	\end{subitems}
  	  \end{resume_subsection}
  
\end{resume_section}

\begin{resume_section}{Work Experience}
\begin{resume_employer}{Amazon Web Services, Inc.}{Software Development Engineer Intern}{Seattle, WA}{Jun. 2023 -- Aug. 2023}
	\item Developed service in Native AWS to export data from an internal multi-primary regionally replicated data store to
	AWS S3 buckets, helping to accelerate org-wide adoption of data store. (Java)
  \item Designed and created operational dashboard to monitor metrics and provide alerts for system failures.
\end{resume_employer}
\begin{resume_employer}{Capital One}{Software Engineer Intern (Technology Internship Program - Center for Machine Learning)}{McLean, VA}{Jun. 2022 -- Aug. 2022}
	\item Designed and implemented dataset access service using Flask and AWS DynamoDB for real-time model serving platform. (Python)
	\item Trained and optimized an XGBoost machine learning model for credit card fraud detection; used Dask for parallel processing of large datasets. 
	\item Deployed model onto production Kubernetes-based platform for use as an highly available API microservice.
\end{resume_employer}
\begin{resume_employer}{University of Maryland}{Teaching Assistant}{College Park, MD}{Aug. 2021 -- Dec. 2022}
  \item Led discussion/lab sections to reinforce class concepts and introduce extra material.
  \item Held office hours; graded projects and exams.
	\item Classes TAed: CMSC216 - Intro. to Computer Systems; CMSC351 - Algorithms; BUFN400 - Intro. to Financial Markets and Financial Datasets
  % - CMSC216 - Intro. to Computer Systems - C, pointers, dynamic memory management, I/O, MIPS Assembly, process control, threads, concurrency, Linux/Unix system.
  
  % - CMSC351 - Algorithms - asymptotic analysis, sorting and searching algorithms, order statistics, graphs and trees, NP-Completeness
\end{resume_employer}
  \begin{resume_employer}{nth Solutions, LLC}{Software Developer Intern}{Exton, PA}{Jan. 2019 -- Jul. 2020}
    \item Developed a user-facing dashboard using Java and JavaFX that facilitates communication of data between an IMU(Accelerometer, Gyroscope, Magnetometer) module and a computer. Dashboard is also responsible for configuring the module for data collection and processing the recorded data.
    \item Implemented a JavaFX Graph / Media Player into the dashboard that enables the visualization of IMU motion data synchronized with recorded video of IMU module's movement.
  \end{resume_employer}
  
\end{resume_section}

\begin{resume_section}{Projects, Activities, and Awards}
  
  \begin{resume_subsection}[(Jan. 2023 - May. 2023)]{Adversarial Communication in Multi-Agent Reinforcement Learning}
    \begin{subitems}
      \item Investigated the effectiveness of attention-based methods for improving total reward
      in Multi-Agent RL settings with benign and adversarial communication among agents.
      \item Designed and implemented approach in PyTorch on top of standard PPO traning algorithm. 
      %\item Presented findings in NeurIPS style report.
    \end{subitems}
  \end{resume_subsection}

 \begin{resume_subsection}[Feb. 2023]{International Collegiate Programming Contest (ICPC)}
   \begin{subitems}
     \item Mid-Atlantic USA Regional - Top 25\%
   \end{subitems}
 \end{resume_subsection}

 \begin{resume_subsection}[]{Lakers Analysis Project}
  \begin{subitems}
  	\item Analyzed historical NBA team data using statistical and data science techniques to explain Lakers' underperformance relative to expectations in 2022.
  	\item Created data visualizations using seaborn and matplotlib to aid in exposition.
  \end{subitems}
  \end{resume_subsection}
  
\end{resume_section}

\begin{resume_section}{Skills}
  \begin{nospacetabbing}
  \textbf{Programming Languages} \= Proficient: Python, C, Java; Familiar: OCaml, C++, Rust, Racket, Scheme, R\\*
  \textbf{Technical Skills}  \> NumPy, Pandas, Scikit-learn, XGBoost, Dask, Matplotlib, Jupyter, MATLAB, \LaTeX \\ *
  \> Kubernetes, Docker, Flask, AWS s3, AWS DynamoDB, JavaFX, VSCode, Git, Agile Dev.
  %\textbf{Test Scores} \> SAT: 1560 (Math: 800, EBRW: 760)\\* %
  %\textbf{Interests} \> Data Analytics, Machine Learning, Consumer Electronics, Design, Computer Software, Stationary\\*

  \end{nospacetabbing}
\end{resume_section}

\end{document}
